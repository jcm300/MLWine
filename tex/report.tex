\documentclass{article}
\usepackage[a4paper, top=3cm, left=3cm, right=2.5cm, bottom=2.5cm]{geometry}
\usepackage[utf8]{inputenc}
\usepackage{graphicx}
\usepackage{float}
\usepackage{fancyvrb}
\usepackage{amsmath}
\usepackage{ragged2e}
\usepackage[portuguese]{babel}

\begin{document}

\title{\Huge
       \textbf{UNIVERSIDADE DO MINHO}\\
       \vspace*{3cm}
       \huge
       \textbf{\textit{Machine Learning de Qualidade de Vinhos}\\
       \vspace*{3cm}
       \large
       Mestrado Integrado de Engenharia Informática\\
       \vspace*{2cm}
       Sistemas de Representação de Conhecimento e Raciocínio\\
       (2ºSemestre/2017-2018)
       \vspace*{\fill}}

\author{\hspace*{-5cm}Número\hspace*{1cm}Nome do(s) Autor(es)\hspace*{\fill}\\
        \hspace*{-5cm}a78468\hspace*{1cm}João Vieira\hspace*{\fill}\\
        \hspace*{-5cm}a78821\hspace*{1cm}José Martins\hspace*{\fill}\\
        \hspace*{-5cm}a77049\hspace*{1cm}Miguel Quaresma\hspace*{\fill}\\
        \hspace*{-5cm}a77689\hspace*{1cm}Simão Barbosa\hspace*{\fill}}

\date{\hspace*{\fill}Braga, Portugal\hspace*{1cm}\\
      \hspace*{\fill}\today\hspace*{1cm}}

\maketitle

\newpage

\justify

\vspace*{\fill}
\section{Resumo}
{\color{red}TODO}
\vspace*{\fill}

\newpage

\vspace*{\fill}
\tableofcontents
\vspace*{\fill}

\newpage

\vspace*{\fill}
\section{Introdução}
{\color{red}TODO}
\vspace*{\fill}

\newpage

\section{Descrição do Trabalho e Análise de Resultados}

Como tinhamos dois datasets sobre vinhos(um de vinho branco e outro de tinto) com as mesmas caracteristicas, juntamos os dois num só, realizando após a junção, um sorteio da ordem de cada linha.
Após isso avaliamos os dados e procedemos à normalização dos mesmos, sendo que o output (quality) foi multiplicado por 0.1 de modo a normalizar.
Realizamos de seguida um teste aos dados de mod a saber quais os valores mais influentes no valor output. Com os resultados decidimos criar as seguintes fórmulas:
\begin{verbatim}
funcao <- quality ~ fixed.acidity+volatile.acidity+citric.acid+residual.sugar+chlorides+free.sulfur.dioxide+total.sulfur.dioxide+density+pH+sulphates+alcohol
funcaoOpt <- quality ~ alcohol+volatile.acidity+sulphates+residual.sugar+total.sulfur.dioxide
funcaoOpt2 <- quality ~ alcohol+volatile.acidity+sulphates
\end{verbatim}
Após isso decidimos dividir os dados em duas partes, uma parte para treinar a rede neuronal (os primeiros 4500) e o resto para testar a rede neuronal treinada.
Por forma a descobrir a melhor rede neuronal, ou seja, que obtém o menor erro com os dados de teste é necessário realizar várias tentativas que foi o que se realizou seguindo a seguinte ordem de execução:
\begin{itemize}
    \item treinar a rede neuronal: rnaWine <- neuralnet( funcao, dadosTreino, lifesign="full", hidden = c(7,5), threshold = 0.01)
    \item preparar os dados de teste: dadosTeste1 <- subset(dadosTeste, select = c("fixed.acidity","volatile.acidity","citric.acid","residual.sugar","chlorides","free.sulfur.dioxide","total.sulfur.dioxide","density","pH","sulphates","alcohol"))
    \item testar a rede: rnaWine1.resultados <- compute(rnaWine1,dadosTeste1)
    \item comparar resultados: resultados1 <- data.frame(atual = dadosTeste$quality, previsao = rnaWine1.resultados$net.result)
    \item arredondar os resultados: resultados1$previsao <- round(resultados1$previsao, digits=1)
    \item calcular o RMSE: rmse(c(dadosTeste$quality),c(resultados1$previsao))
\end{itemize}
O arredondamento dos resultados é realizado com 1 casa decimal devido que o output dos datasets não é de 0 ou 1(falso ou verdadeiro) mas sim de 0 a 10(ou após normalizados de 0 a 1, de 0.1 em 0.1).
É também importante referir que como variamos a fórmula entre as várias tentativas na preperação de dados de teste os dados definidos mudam consoante a fórmula. É importante referir também que durante os testes é variado o hidden e o threshold da rede neuronal como óbvio de modo a obter melhores resultados.
O hidden é referente às camadas intermédias da rede neuronal enquanto que a fórmula inserida define os neurónios de entrada e de saída.

\subsection{Tentativas}

\subsubsection{threshold=0.1}
\subsubsection{Com funcao}
Rede Neuronal : Erro
11 -> 6 -> 1 : 0.07857646844
11 -> 3 -> 3 -> 1 : 0.07777579064
11 -> 10 -> 8 -> 1 : 0.07716171911
11 -> 7 -> 4 -> 1 : 0.07611629613

\subsubsection{Com funcaoOpt}
Rede Neuronal : Erro
5 -> 5 -> 3 -> 1 : 0.08027868335

\subsubsection{Com funcaoOpt2}
Rede Neuronal : Erro
3 -> 6 -> 2 : 0.07958958319

\subsubsection{threshold=0.01}
\subsubsection{Com funcao}
Rede Neuronal : Erro
11 -> 6 -> 3 -> 1 : 0.07598460693
11 -> 7 -> 4 -> 1 : 0.07738852289
11 -> 12 -> 4 -> 1 : 0.07732378969
11 -> 5 -> 3 -> 1 : 0.0760833952
11 -> 10 -> 1 : 0.07528945978
11 -> 8 -> 7 -> 1 : 0.09155639716
11 -> 7 -> 5 -> 1 : 0.07595164896

\subsubsection{Com funcaoOpt}
Rede Neuronal : Erro
5 -> 6 -> 2 -> 1 : 0.07628058791
5 -> 8 -> 4 -> 1 : 0.07761466391
5 -> 4 -> 1 : 0.07860832592
5 -> 6 -> 3 -> 1 : 0.07719416045
5 -> 6 -> 5 -> 1 : 0.07780797595
5 -> 4 -> 3 -> 1 : 0.07784014795
5 -> 4 -> 2 -> 1 : 0.07738852289
5 -> 3 -> 2 -> 1 : 0.07835310341
5 -> 4 -> 5 -> 4 -> 1 : 0.07706431316

\subsubsection{Com funcaoOpt2}
Rede Neuronal : Erro
3 -> 4 -> 1 : 0.07946365023
3 -> 2 -> 1 : 0.07999749621
3 -> 2 -> 2 -> 1 : 0.07930595281
3 -> 3 -> 2 -> 1 : 0.07984085322
3 -> 4 -> 3 -> 1 : 0.07971531721
3 -> 4 -> 4 -> 1 : 0.0796839023
3 -> 5 -> 1 : 0.0796839023
3 -> 5 -> 3 -> 1 : 0.07996619216
3 -> 2 -> 3 -> 2 -> 1 : 0.07930595281
3 -> 4 -> 5 -> 4 -> 1 : 0.08015383307

O menor foi obtido claro usando todas as caracteristicas dos vinhos contudo o erro não aumenta assim tão significantemente se se usar apenas as 5 principais caracteristicas. Mesmo com apenas 3 o erro não é assim tão diferente.

{\color{red}TODO}

\newpage

\vspace*{\fill}
\section{Conclusões e Sugestões}
{\color{red}TODO}
\vspace*{\fill}

\newpage

\end{document}
